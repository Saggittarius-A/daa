\documentclass[12pt]{report}
\usepackage{amsmath}
\usepackage{latexsym}
\usepackage{amsfonts}
\usepackage[normalem]{ulem}
\usepackage{soul}
\usepackage{array}
\usepackage{amssymb}
\usepackage{extarrows}
\usepackage{graphicx}
\usepackage[backend=biber,
style=numeric,
sorting=none,
isbn=false,
doi=false,
url=false,
]{biblatex}\addbibresource{bibliography.bib}

\usepackage{subfig}
\usepackage{wrapfig}
\usepackage{wasysym}
\usepackage{enumitem}
\usepackage{adjustbox}
\usepackage{ragged2e}
\usepackage[svgnames,table]{xcolor}
\usepackage{tikz}
\usepackage{longtable}
\usepackage{changepage}
\usepackage{setspace}
\usepackage{hhline}
\usepackage{multicol}
\usepackage{tabto}
\usepackage{float}
\usepackage{multirow}
\usepackage{makecell}
\usepackage{fancyhdr}
\usepackage[toc,page]{appendix}
\usepackage[hidelinks]{hyperref}
\usetikzlibrary{shapes.symbols,shapes.geometric,shadows,arrows.meta}
\tikzset{>={Latex[width=1.5mm,length=2mm]}}
\usepackage{flowchart}\usepackage[paperheight=11.0in,paperwidth=8.5in,left=1.0in,right=1.0in,top=1.0in,bottom=1.0in,headheight=1in]{geometry}
\usepackage[utf8]{inputenc}
\usepackage[T1]{fontenc}
\TabPositions{0.5in,1.0in,1.5in,2.0in,2.5in,3.0in,3.5in,4.0in,4.5in,5.0in,5.5in,6.0in,}

\urlstyle{same}

\renewcommand{\_}{\kern-1.5pt\textunderscore\kern-1.5pt}
\setcounter{tocdepth}{5}
\setcounter{secnumdepth}{5}

\setlistdepth{9}
\renewlist{enumerate}{enumerate}{9}
		\setlist[enumerate,1]{label=\arabic*)}
		\setlist[enumerate,2]{label=\alph*)}
		\setlist[enumerate,3]{label=(\roman*)}
		\setlist[enumerate,4]{label=(\arabic*)}
		\setlist[enumerate,5]{label=(\Alph*)}
		\setlist[enumerate,6]{label=(\Roman*)}
		\setlist[enumerate,7]{label=\arabic*}
		\setlist[enumerate,8]{label=\alph*}
		\setlist[enumerate,9]{label=\roman*}

\renewlist{itemize}{itemize}{9}
		\setlist[itemize]{label=$\cdot$}
		\setlist[itemize,1]{label=\textbullet}
		\setlist[itemize,2]{label=$\circ$}
		\setlist[itemize,3]{label=$\ast$}
		\setlist[itemize,4]{label=$\dagger$}
		\setlist[itemize,5]{label=$\triangleright$}
		\setlist[itemize,6]{label=$\bigstar$}
		\setlist[itemize,7]{label=$\blacklozenge$}
		\setlist[itemize,8]{label=$\prime$}

\setlength{\topsep}{0pt}\setlength{\parindent}{0pt}

\renewcommand{\arraystretch}{1.3}

\begin{document}
\begin{Center}
{\fontsize{14pt}{16.8pt}\selectfont \textbf{To create a matrix of size 50$ \times $ 50 of numbers ranging from 0 to 9.}\par}
\end{Center}\par

\begin{Center}
{\fontsize{14pt}{16.8pt}\selectfont \textbf{Find the length of the largest sorted component horizontally.}\par}
\end{Center}\par

\begin{Center}
Akash Anand, Parth Kataria, Shruti Nanda
\end{Center}\par

\begin{Center}
\href{mailto:iit2019015@iiita.ac.in}{{\fontsize{13pt}{15.6pt}\selectfont \textcolor[HTML]{1155CC}{\ul{iit2019015@iiita.ac.in}}}, \href{mailto:iit20190016@iiita.ac.in}{\textcolor[HTML]{1155CC}{\ul{iit20190016@iiita.ac.in}}}, \href{mailto:iit2019017@iiita.ac.in}{\textcolor[HTML]{1155CC}{\ul{iit2019017@iiita.ac.in}}}\par}
\end{Center}\par

\begin{Center}
\textit{IV semester,Department of Information Technology}
\end{Center}\par

\begin{Center}
\textit{Indian Institute of Information Technology, Allahabad}
\end{Center}\par


\vspace{\baselineskip}
\begin{multicols}{2}

\vspace{\baselineskip}
\textbf{\textit{Abstract-This\ paper contains the algorithms to find the length of the largest sorted component horizontally  in a 2-D matrix. We will discuss time complexities of all algorithms and differences in them.}}\par


\vspace{\baselineskip}

\vspace{\baselineskip}
\begin{Center}
\textbf{1. INTRODUCTION}
\end{Center}\par

Given a 2d array with n number of rows and m number of columns where n and m less than 50.\par

The elements of the array will range from 0 to 9.\par

We will print the largest sorted component in the array horizontally.\par

This report further contains:\par

\begin{enumerate}
	\item Algorithm Design and Analysis\par

	\item Experimental study and complexity.\par

	\item Conclusion
\end{enumerate}\par


\vspace{\baselineskip}
\begin{Center}
\textbf{2. ALGORITHM DESIGN}
\end{Center}\par

We designed three algorithms to find the largest sorted array horizontally in 2-D matrix. Basic steps for designing the algorithm are-\par

\begin{enumerate}
	\item Create a 2-d array of size(m$\ast$ n), where (m,n<50)\par

	\item Take the elements of array as user input.\par

	\item Now pass the array along with the size as arguments in the functions of following algorithms to print the length of the largest sorted array horizontally.\par


\vspace{\baselineskip}
Algorithm 1-\par

int Brute(int a[][], int n, int m)\par

$ \{ $ \par

\ \ \ \ int\ max\_cnt=0,cnt=1;\    \par

\ \ \ \ \ \ \ \ for(i =0 to  n)$ \{ $ \par

\ \ \  \tab \par

for( j=1 to j=m)\par

\ \ \  \tab $ \{ $ \par

\ \ \  \tab \ \ \  if(a[i][j]>a[i][j-1])\par

\ \ \  \tab \ \ \ \ \ \  cnt++;\par

\ \ \  \tab \ \  else$ \{ $ \par

\ \ \  \tab \ \ \  max\_cnt=max(cnt,max\_cnt);\par

\ \ \  \tab \ \ \  cnt=1;$ \} $ \par

\tab $ \} $ \par

Return max\_cnt;\par

$ \} $ \par


\vspace{\baselineskip}
Analysis: In this algorithm we have checked for each row if the array is increasing or decreasing we have increased the cnt variable and else we have updated the max\_cnt variable and make the cnt variable 0.\par


\vspace{\baselineskip}
Algorithm 2-\par

int sorting\_cnt(int a[][], int n, int m)\par

$ \{ $ \par

\ \ \  int n, m;\par

\ \ \  int max\_cnt=0,cnt=1;\par

\ \ \  vector<int>count;\par

\ \ \  for( i=0 to i=n)$ \{ $ \par

\ \ \  for( j=1 to m)\par

\ \ \  $ \{ $ \par

\ \ \ \ \ \  if(a[i][j]>a[i][j-1])$ \{ $ \par

\tab cnt++;\tab \par

\tab v.push\_back(cnt);\par

\ \ \ \  $ \} $ \par

\ \ \  else\par

\ \ \  \tab cnt=1;\par

\ \ \  $ \} $ $ \} $ \par

\ \ \  sort(count.begin(), count.end());\par

\tab Return count[count.size()-1];\par

$ \} $ \par

Analysis:\par

In\ this algorithm we are storing the  length of the sorted subarrays of the array in a vector count using nested for loop . Then we are sorting it printing the largest length of the sorted subsequence. \par


\vspace{\baselineskip}

\vspace{\baselineskip}
\begin{Center}
\textbf{3. EXPERIMENTAL STUDY AND ANALYSIS}
\end{Center}\par


\vspace{\baselineskip}
A.brute()\par


\vspace{\baselineskip}
The Number of iteration in first for loop is the no. of rows present in the array and in the second for loop, number of iteration is equal to the number of columns in array. So, the complexity will be O(n$\ast$ m).\par

\textbf{t }{\fontsize{7pt}{8.4pt}\selectfont \textbf{worst\par}} : O(n+n$\ast$ m).\par


\vspace{\baselineskip}
B.sorting\_cnt()\par

In this algorithm we are storing the length of each sorted subsequence in the complexity of o(n$ \string^ $ 2) using a for loop. Then we are sorting it in the complexity of O(n$ \string^ $ 2logn). So, the overall complexity will be:\par

\textbf{t }{\fontsize{7pt}{8.4pt}\selectfont \textbf{worst\par}} : O(n$ \string^ $ 2logn).\par


\vspace{\baselineskip}

\vspace{\baselineskip}

\begin{table}[H]
 			\centering
\begin{tabular}{p{0.13in}p{0.24in}p{0.26in}p{0.68in}p{0.69in}}
\hline
%row no:1
\multicolumn{1}{|p{0.13in}}{{\fontsize{10pt}{12.0pt}\selectfont S no.}} & 
\multicolumn{1}{|p{0.24in}}{{\fontsize{10pt}{12.0pt}\selectfont N}} & 
\multicolumn{1}{|p{0.26in}}{{\fontsize{10pt}{12.0pt}\selectfont M}} & 
\multicolumn{1}{|p{0.68in}}{{\fontsize{10pt}{12.0pt}\selectfont Algorithm1}} & 
\multicolumn{1}{|p{0.69in}|}{{\fontsize{10pt}{12.0pt}\selectfont Algorithm2}} \\
\hhline{-----}
%row no:2
\multicolumn{1}{|p{0.13in}}{{\fontsize{10pt}{12.0pt}\selectfont 1}} & 
\multicolumn{1}{|p{0.24in}}{{\fontsize{10pt}{12.0pt}\selectfont 5}} & 
\multicolumn{1}{|p{0.26in}}{{\fontsize{10pt}{12.0pt}\selectfont 5}} & 
\multicolumn{1}{|p{0.68in}}{{\fontsize{10pt}{12.0pt}\selectfont 30}} & 
\multicolumn{1}{|p{0.69in}|}{{\fontsize{10pt}{12.0pt}\selectfont 25.219280}} \\
\hhline{-----}
%row no:3
\multicolumn{1}{|p{0.13in}}{{\fontsize{10pt}{12.0pt}\selectfont 2}} & 
\multicolumn{1}{|p{0.24in}}{{\fontsize{10pt}{12.0pt}\selectfont 20}} & 
\multicolumn{1}{|p{0.26in}}{{\fontsize{10pt}{12.0pt}\selectfont 20}} & 
\multicolumn{1}{|p{0.68in}}{{\fontsize{10pt}{12.0pt}\selectfont 420}} & 
\multicolumn{1}{|p{0.69in}|}{{\fontsize{10pt}{12.0pt}\selectfont 800.87712}} \\
\hhline{-----}
%row no:4
\multicolumn{1}{|p{0.13in}}{{\fontsize{10pt}{12.0pt}\selectfont 3}} & 
\multicolumn{1}{|p{0.24in}}{{\fontsize{10pt}{12.0pt}\selectfont 40}} & 
\multicolumn{1}{|p{0.26in}}{{\fontsize{10pt}{12.0pt}\selectfont 40}} & 
\multicolumn{1}{|p{0.68in}}{{\fontsize{10pt}{12.0pt}\selectfont 1640}} & 
\multicolumn{1}{|p{0.69in}|}{{\fontsize{10pt}{12.0pt}\selectfont 1678.75424}} \\
\hhline{-----}
%row no:5
\multicolumn{1}{|p{0.13in}}{{\fontsize{10pt}{12.0pt}\selectfont 4}} & 
\multicolumn{1}{|p{0.24in}}{{\fontsize{10pt}{12.0pt}\selectfont 50}} & 
\multicolumn{1}{|p{0.26in}}{{\fontsize{10pt}{12.0pt}\selectfont 50}} & 
\multicolumn{1}{|p{0.68in}}{{\fontsize{10pt}{12.0pt}\selectfont 2550}} & 
\multicolumn{1}{|p{0.69in}|}{{\fontsize{10pt}{12.0pt}\selectfont 6687.38561}} \\
\hhline{-----}
%row no:6
\multicolumn{1}{|p{0.13in}}{{\fontsize{10pt}{12.0pt}\selectfont 5}} & 
\multicolumn{1}{|p{0.24in}}{{\fontsize{10pt}{12.0pt}\selectfont 70}} & 
\multicolumn{1}{|p{0.26in}}{{\fontsize{10pt}{12.0pt}\selectfont 70}} & 
\multicolumn{1}{|p{0.68in}}{{\fontsize{10pt}{12.0pt}\selectfont 4970}} & 
\multicolumn{1}{|p{0.69in}|}{{\fontsize{10pt}{12.0pt}\selectfont 10067.9962}} \\
\hhline{-----}
%row no:7
\multicolumn{1}{|p{0.13in}}{{\fontsize{10pt}{12.0pt}\selectfont 6}} & 
\multicolumn{1}{|p{0.24in}}{{\fontsize{10pt}{12.0pt}\selectfont 100}} & 
\multicolumn{1}{|p{0.26in}}{{\fontsize{10pt}{12.0pt}\selectfont 100}} & 
\multicolumn{1}{|p{0.68in}}{{\fontsize{10pt}{12.0pt}\selectfont 10100}} & 
\multicolumn{1}{|p{0.69in}|}{{\fontsize{10pt}{12.0pt}\selectfont 43000.7712}} \\
\hhline{-----}
%row no:8
\multicolumn{1}{|p{0.13in}}{{\fontsize{10pt}{12.0pt}\selectfont 7}} & 
\multicolumn{1}{|p{0.24in}}{{\fontsize{10pt}{12.0pt}\selectfont 1000}} & 
\multicolumn{1}{|p{0.26in}}{{\fontsize{10pt}{12.0pt}\selectfont 1000}} & 
\multicolumn{1}{|p{0.68in}}{{\fontsize{10pt}{12.0pt}\selectfont 1001000}} & 
\multicolumn{1}{|p{0.69in}|}{{\fontsize{10pt}{12.0pt}\selectfont 600000.568}} \\
\hhline{-----}
%row no:9
\multicolumn{1}{|p{0.13in}}{{\fontsize{10pt}{12.0pt}\selectfont 8}} & 
\multicolumn{1}{|p{0.24in}}{{\fontsize{10pt}{12.0pt}\selectfont 1000}} & 
\multicolumn{1}{|p{0.26in}}{{\fontsize{10pt}{12.0pt}\selectfont 10000}} & 
\multicolumn{1}{|p{0.68in}}{{\fontsize{10pt}{12.0pt}\selectfont 10001000}} & 
\multicolumn{1}{|p{0.69in}|}{{\fontsize{10pt}{12.0pt}\selectfont 70000000}} \\
\hhline{-----}

\end{tabular}
 \end{table}

\vspace{\baselineskip}
	\item \textbf{TIME COMPLEXITY}
\end{enumerate}\par

The algorithms were tested against positive random sets of variable sizes.The result thus obtained from this experiment is given below:\par


\vspace{\baselineskip}
For brute():\par

\begin{FlushLeft}
Best case : $ \Omega $ (n$ \string^ $ 2)
\end{FlushLeft}\par

\begin{FlushLeft}
Average case : $ \theta $ (n$ \string^ $ 2)
\end{FlushLeft}\par

\begin{FlushLeft}
Worst case : O(n$ \string^ $ 2)
\end{FlushLeft}\par


\vspace{\baselineskip}
For sorting\_cnt():\par

Best case : $ \Omega $ (n$ \string^ $ 2)\par

Average case : $ \theta $ (n$ \string^ $ 2.logn)\par

Worst case : O(n$ \string^ $ 2logn)\par


\vspace{\baselineskip}


\begin{figure}[H]
	\begin{FlushLeft}		\includegraphics[width=3.0in,height=2.4in]{./media/image1.png}
	\end{FlushLeft}\end{figure}

\par


\vspace{\baselineskip}


\begin{figure}[H]
	\begin{FlushLeft}		\includegraphics[width=3.0in,height=2.4in]{./media/image2.png}
	\end{FlushLeft}\end{figure}


\par

\begin{figure}[H]
	\begin{FlushLeft}		\includegraphics[width=3.0in,height=2.4in]{./media/image3.png}
	\end{FlushLeft}\end{figure}

\par


\vspace{\baselineskip}

\vspace{\baselineskip}

\vspace{\baselineskip}
\begin{FlushLeft}
\textbf{\ \ \ \ \ \ \ \ \ \  5.CONCLUSION}
\end{FlushLeft}\par

\begin{FlushLeft}
Time complexity for finding the length of the largest sorted subarray will be minimum in brute force which is O(n$ \string^ $ 2) and for sorting count algorithm time complexity will be O(n$ \string^ $ 2logn ).
\end{FlushLeft}\par


\vspace{\baselineskip}

\vspace{\baselineskip}
\begin{Center}
\textbf{\ \ \ \  6. REFERENCES}
\end{Center}\par

\begin{enumerate}
	\item \href{https://www.quora.com/How-would-one-use-Arrays-sort-on-a-multidimensional-array-of-ints-by-the-first-element-of-each-sub-array-in-Java}{\textcolor[HTML]{1155CC}{\ul{https://www.quora.com/How-would-one-use-Arrays-sort-on-a-multidimensional-array-of-ints-by-the-first-element-of-each-sub-array-in-Java}}}\par

	\item https://www.geeksforgeeks.org/longest-increasing-path-matrix/
\end{enumerate}\par


\vspace{\baselineskip}

\vspace{\baselineskip}
\tab 
\vspace{\baselineskip}
\vspace{\baselineskip}

\vspace{\baselineskip}

\vspace{\baselineskip}

\vspace{\baselineskip}

\vspace{\baselineskip}

\vspace{\baselineskip}

\vspace{\baselineskip}

\vspace{\baselineskip}

\vspace{\baselineskip}

\vspace{\baselineskip}

\end{multicols}

\printbibliography
\end{document}